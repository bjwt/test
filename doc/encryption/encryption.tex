\documentclass{article}
\usepackage{graphicx}
\usepackage{float}
\usepackage[dvipsnames]{xcolor}
\usepackage{listings}
\usepackage{indentfirst}
\usepackage{paralist}

\lstset{numbers=left,framexleftmargin=10mm,frame=none,backgroundcolor=\color[RGB]{245,245,244},keywordstyle=\bf\color{blue},identifierstyle=\bf,numberstyle=\color[RGB]{0,192,192},commentstyle=\it\color[RGB]{0,96,96},stringstyle=\rmfamily\slshape\color[RGB]{128,0,0},showstringspaces=false}
\begin{document}
\title{\textbf{Encrypting Swap investigation report}}
%\author{Tao Wu}
%\date{\today}
\maketitle
\section{Introduction}
There are two schemes of encrypting swap partition:\\

A: Encrypt the partition underlying the file system, which works at the block
level. This can be done by using some libs/utilities such as blivet. Through
this we can protect sensitive data on the disks which have been swapped from the
memory. Once a disk is encrypted in this method, it will require the user to
input the passphrase when system boots to decrypt the disk.\\

B: Encrypt the partition by kernel during running time. In this scheme, the
cycle of encryption begins when system boots and ends when system shuts down.
This can prevent some user processes to access sensitive data on swap partition
which they have no authority. In this method we can set a random passphrase
generated from /dev/urandom.\\

Scheme B is which we are seeking to implement and the following is a verified
solution for scheme B.\\

Section 4 gives a solution for scheme A.

\section{Example of the solution for B:}

\textbf{Steps:}\\

1. Have a partition prepared.

2. Configure /etc/fstab.

\qquad \underline{	/dev/mapper/\textbf{abc}\quad		swap\quad	swap\quad defaults\quad	0\quad 0}

3. Configure /etc/crypttab.

\qquad \underline{	\textbf{abc}\quad	\textbf{/dev/sdb2}\quad	/dev/urandom\quad	swap}

4. Reboots. \\

\textbf{Explanation:}\\

There is no extra requirement to the partition, meaning that it does not even
need to be set as 'swap' when it is created. So this solution has nothing to do
with blivet. Any existing partition can be used here, but once it is configured
as above, any data on the partition will be destroyed when system reboots.\\

In these two configure files, you change only two strings: ``abc'' and
``/dev/sdb2''.\\

1. Replace ``abc'' with any regular string, just making sure the consistency
between the two files.\\

2. Replace the ``/dev/sdb2'' with the path of the partition which you want to use
as an encrypted swap.

\section{How to verify the result?}

\begin{itemize}
\item \textbf{Check if it is used as a swap}\\

Use `swapon -s' to list all active swaps, the output is as follows:\\

\underline{	/dev/dm-2		partition	102396	0	-1}\\

Use `ls -l /dev/mapper' to locate the path of the swap, as:\\

\underline{	lrwxrwxrwx. 1 root root       7 Apr 22 02:10 abc -$>$ ../dm-2}\\

From this we can confirm that /dev/mapper/abc (/dev/dm-2) is now a swap.\\

\item \textbf{Check if it is encrypted}\\

Use `dmsetup ls --target=crypt' to list all encrypted devices, as:\\

\underline{	abc	(253, 2)}\\

From this we can confirm that `abc' is an encrypted device.
\end{itemize}

\section{Solution for scheme A}
\begin{lstlisting}[language=python]
import blivet

b=blivet.Blivet()
b.reset()

sdd=b.devicetree.getDeviceByName(``sdd'')
disks=[sdd]

for i in disks:
  b.recursiveRemove(i)

for i in disks:
  b.initializeDisk(i)

# Create a partition for use
factory = blivet.devicefactory.PartitionFactory(
          b, 350, disks, fstype=``swap'',
          encrypted = True, name=``test'' )
factory.configure()

# Set passphrase and map name for encrypted partitions
for device in b.devices:
    if device.format.type ==  ``luks'' and 
        not device.format.exists:
        if not device.format.hasKey:
            device.format.passphrase = ``123456''
            device.format.mapName = ``encrypt''

b.doIt()
\end{lstlisting}

\section{Reference}

Running `man 5 crypttab' on linux is a good reference.
\end{document}
